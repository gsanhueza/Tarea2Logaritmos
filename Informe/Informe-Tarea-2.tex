\documentclass[letterpaper,10pt]{article}

\usepackage[utf8]{inputenc}
\usepackage[spanish]{babel}
\usepackage{fontenc}
\usepackage[dvipdfmx]{graphicx}
\usepackage{bmpsize,wrapfig,xcolor}
\usepackage{fullpage}
\usepackage{amssymb}
\usepackage[hidelinks]{hyperref}

% Para evitar que se indente solo a cada rato
\setlength\parindent{0pt}

\begin{document}
	\begin{titlepage}

		\begin{wrapfigure}{R}{0.3\textwidth}
			\includegraphics[width=0.3\textwidth]{logoFCFM.png}
		\end{wrapfigure}

		\noindent \phantom - % "Hax" para que quede alineada la imagen con el texto

		Universidad de Chile

		Facultad de Ciencias Físicas y Matemáticas

		Depto. de Ciencias de la Computación

		CC4102 - Diseño y Análisis de Algoritmos

		\vfill

		\begin{center}
			\begin{Huge}
				{\textbf{Tarea 2}}
			\end{Huge}
		\end{center}

		\vfill

		\begin{flushright}
			\begin{tabular}{lll}
				Integrantes	&:	& Rodrigo Delgado\\
						&	& Belisario Panay\\
						&	& Gabriel Sanhueza\\
				Profesor	&:	& Gonzalo Navarro\\
				Ayudantes	&:	& Sebastián Ferrada\\
						&	& Willy Maikowski\\
				Auxiliar	&:	& Jorge Bahamondes\\
			\end{tabular}
		\end{flushright}

	\end{titlepage}

	% % % % % % % % % % % % % % % % % % % % % % % % % % % % % % % % % % % % % % % % % % % % % % % % % % % % % % % % % % % % % % % % % % % % % % % % % % % % % % % % % % % % % % % % % %
	\newpage
	% % % % % % % % % % % % % % % % % % % % % % % % % % % % % % % % % % % % % % % % % % % % % % % % % % % % % % % % % % % % % % % % % % % % % % % % % % % % % % % % % % % % % % % % % %

	\tableofcontents

	% % % % % % % % % % % % % % % % % % % % % % % % % % % % % % % % % % % % % % % % % % % % % % % % % % % % % % % % % % % % % % % % % % % % % % % % % % % % % % % % % % % % % % % % % %
	\newpage
	% % % % % % % % % % % % % % % % % % % % % % % % % % % % % % % % % % % % % % % % % % % % % % % % % % % % % % % % % % % % % % % % % % % % % % % % % % % % % % % % % % % % % % % % % %

	\section{Introducción}

	Se pide que plantee una hipótesis con respecto al tiempo amortizado de construcción de una estructura de este
	tipo y al tiempo de búsqueda, y la ponga a prueba de forma experimental.
	Se espera que se implemente la estructura y los algoritmos correspondientes, y se entregue un informe.
	En el presente informe se muestra el diseño, implementación y experimentación para resolver la Tarea 2 del curso CC4102 Diseño y Análisis de Algoritmos de la carrera de ingeniería civil en Computación de la Universidad de Chile. 
	La Tarea consiste en crear un SuffixTree en tiempo lineal, donde se entiende por Suffix Tree a una estructura de datos del tipo Arbol, la cual esta formada por Nodos que poseen información interna y referencias a sus hijo. En particular el SuffixTree almacena los sufijos de un string en forma de arbol, de modo que cada arco contiene los caracteres para formar una palabra, si se llega a una hoja ( nodo terminal ) es porque recorrimos los arcos necesarios para formar un sufijo. La idea de la tarea es que el algoritmo a desarrollar es on-line, de orden lineal y con una implementación mas sencilla con respecto a algoritmos similares ( algoritmo de Wiener y algoritmo de McCreight ). 
	
	\subsection{Problema a resolver}

	\subsection{Hipótesis}

	\subsection*{Especificaciones de la máquina utilizada}

	% % % % % % % % % % % % % % % % % % % % % % % % % % % % % % % % % % % % % % % % % % % % % % % % % % % % % % % % % % % % % % % % % % % % % % % % % % % % % % % % % % % % % % % % % %
	\newpage
	% % % % % % % % % % % % % % % % % % % % % % % % % % % % % % % % % % % % % % % % % % % % % % % % % % % % % % % % % % % % % % % % % % % % % % % % % % % % % % % % % % % % % % % % % %


	\section{Diseño Experimental}

	\subsection{Metodología}

	\subsection{Structs}

	\subsubsection{Rectangle}

	\subsubsection{Node}

	\subsection{Constantes}

	\subsection{Funciones}

	% % % % % % % % % % % % % % % % % % % % % % % % % % % % % % % % % % % % % % % % % % % % % % % % % % % % % % % % % % % % % % % % % % % % % % % % % % % % % % % % % % % % % % % % % %
	\newpage
	% % % % % % % % % % % % % % % % % % % % % % % % % % % % % % % % % % % % % % % % % % % % % % % % % % % % % % % % % % % % % % % % % % % % % % % % % % % % % % % % % % % % % % % % % %

	\section{Presentación de los Resultados}

	\subsection{Tiempo de Inserción del R-Tree}

	% % % % % % % % % % % % % % % % % % % % % % % % % % % % % % % % % % % % % % % % % % % % % % % % % % % % % % % % % % % % % % % % % % % % % % % % % % % % % % % % % % % % % % % % % %
	\newpage
	% % % % % % % % % % % % % % % % % % % % % % % % % % % % % % % % % % % % % % % % % % % % % % % % % % % % % % % % % % % % % % % % % % % % % % % % % % % % % % % % % % % % % % % % % %

	\subsection{Espacio ocupado y porcentaje de llenado de páginas de disco}

	% % % % % % % % % % % % % % % % % % % % % % % % % % % % % % % % % % % % % % % % % % % % % % % % % % % % % % % % % % % % % % % % % % % % % % % % % % % % % % % % % % % % % % % % % %
	\newpage
	% % % % % % % % % % % % % % % % % % % % % % % % % % % % % % % % % % % % % % % % % % % % % % % % % % % % % % % % % % % % % % % % % % % % % % % % % % % % % % % % % % % % % % % % % %

	\subsection{Desempeño de operación \textit{Buscar}}

	% % % % % % % % % % % % % % % % % % % % % % % % % % % % % % % % % % % % % % % % % % % % % % % % % % % % % % % % % % % % % % % % % % % % % % % % % % % % % % % % % % % % % % % % % %
	\newpage
	% % % % % % % % % % % % % % % % % % % % % % % % % % % % % % % % % % % % % % % % % % % % % % % % % % % % % % % % % % % % % % % % % % % % % % % % % % % % % % % % % % % % % % % % % %

	\section{Análisis y Conclusiones}

	\subsection{Control de Overflow}

	\subsection{Buscar}

	\subsection{Conclusiones}

\end{document}
